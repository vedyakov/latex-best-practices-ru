% !TeX encoding = UTF-8
\documentclass[12pt, a4paper]{article}
\usepackage[utf8]{inputenc}
\usepackage[english, russian]{babel}

% for indent first paragraph within section
\usepackage{indentfirst}

%for text boundaries
\usepackage[a4paper, left=15mm, right=15mm, top=15mm, bottom=15mm]{geometry}

%for some mathematical things
\usepackage{amsmath}

\PassOptionsToPackage{usenames,dvipsnames}{xcolor}
\usepackage[listings,most]{tcolorbox}

\lstdefinestyle{mystyle}{language=[LaTeX]TeX,
	basicstyle=\ttfamily\small,
	texcsstyle=*\color{BlueViolet},
	breaklines=true,
	columns=fullflexible,
	tabsize=2
}
\newtcblisting{example}{
	width=\linewidth, sharp corners,
	sidebyside,%
	lower separated=true,
	listing options={style=mystyle},
	colframe=PineGreen,
	colback=white,
	before lower={\vspace{-2em}}
}
\newtcblisting{wrongexample}{
	width=\linewidth, sharp corners,
	sidebyside,%
	lower separated=true,
	listing options={style=mystyle},
	colframe=Mahogany,
	colback=white,
	before lower={\vspace{-2em}}
}
\newtcblisting{listingonly}{
	width=\linewidth, sharp corners,
	listing only,%
	listing options={style=mystyle},
	colframe=Mahogany,
	colback=white
}

\title{{\LaTeX} Best Practices}
\author{Алексей Ведяков}
\date{2020}

\begin{document}

\maketitle
 
Ниже изложены основные практики, которые настоятельно рекомендуются к
применению, а в некоторых случаях являются обязательными. Их применение
обусловлено основами типографики, традициями сложившимися в русской и
англоязычной академической среде, здравым смыслом, опытом работы в TeX, в том числе совместной, и, конечно, в некоторой степени субъективными причинами.

\section{Оформление формул}

\subsection{Окружение}

Все блочные формулы только в окружении \texttt{align}. Без
\texttt{aligned}, \texttt{equation}, \texttt{\$\$\ldots\$\$} и так далее, и уже
тем более без вложенного \texttt{aligned} в \texttt{align}.

\textbf{Причина:} \texttt{align} --~самое мощное окружение, другие нет
причин использовать, а унификация помогает сделать оформление формул в
документе единообразным, уменьшить проблемы при совместной работе.

\subsection{Нумерация формул}

% TODO блочные формулы
\begin{enumerate}
	\item Для выравнивания многострочных формул используется \& (амперсанд)
	\begin{enumerate}
		\item Знак ставится до математического знака, иначе оно <<прилипает>> к стоящему справа выражению:
\begin{wrongexample}
\begin{align}
	y(t) = & a_1 x(t) + b_1.
\end{align}
\end{wrongexample}
\begin{example}
\begin{align}
	y(t) & = a_1 x(t) + b_1.
\end{align}
\end{example}	

		
		\item Строки формул выравниваются по знаку равно:
\begin{example}
\begin{align}
	y(t) & = a_1 x(t) + b_1,
	\\
	z(t) & = a_2 x(t) + b_2.
\end{align}
\end{example}
\begin{example}
\begin{align}
	\cos 2a & = \cos a - \sin a
		\nonumber \\
		& = 2\cos a - 1
\end{align}
\end{example}
			
			\item Если последующая строка при переносе начинается не со знака равно, то её содержимое рекомендуется подвинуть правее с помощью команды \verb|\quad|:
\begin{example}
\begin{align}
	\cos 2a & = \cos a -
		\nonumber \\
		& \quad - \sin a
		\nonumber \\
		& = 2\cos a - 1.
\end{align}
\end{example}
			Так визуально проще отделять шаги вычислений друг от друга.
	\end{enumerate}
	\item Все отдельные математические выражения в блочных формулах нумеруются (автоматически). Вручную нумерация убирается (с помощью команды \verb|\nonumber|) только для строк, которые являются промежуточными в вычислениях, при этом нумерация последняя строка в вычислениях.
\begin{example}
\begin{align}
	\cos 2a & = \cos a - \sin a
		\nonumber \\
		& = 2\cos a - 1.
\end{align}
\end{example}
	Если требуется не нумеровать те формулы, на которые нет ссылок, то это делается автоматически:
\begin{listingonly}
	\usepackage{mathtools}
	\mathtoolsset{showonlyrefs}
\end{listingonly}
	В 99\% случаев, это будет намного лучше, чем вручную нумеровать и не нумеровать формулы по неясной логике, расставляя везде \verb|\nonumber| или используя окружение \texttt{align}, что заставит страдать всех тех, кому все же понадобится сделать ссылку. Даже, если честно оставлять нумерованными только те формулы, на которые есть ссылка, то с течением времени это становится трудно поддерживать. Ссылки на формулы появляются позже и зачастую много позже в тексте, а значит сначала везде будет расставлено \verb|\nonumber|, которое надо будет убирать по мере написания текста. Кроме того, при удалении ссылок на формулы из текста, если делать вручную, то надо будет опять идти и расставлять \verb|\nonumber|. Используя же пакет, можно быть уверенным, что нумерованных формул без ссылок не останется. При совместной работе следить за правильной нумерацией еще сложнее. Можно расставлять везде \verb|\nonumber|, где нет \verb|\label|, но \verb|\label| могли просто забыть убрать при удалении ссылки на него из текста.
\end{enumerate}

%\item Перенос математического выражения осуществляется по математическому знаку.
% TODO: $\times$
% TODO повтор знака в русской традиции
% TODO label после caption
%\item Все формулы и строки в формулах нумерованные, вручную нумерация убирается только с помощью команды \nonumber для строк, которые являются промежуточными в вычислениях, а нумерация последняя строка.
%\begin{example}
%\begin{align}
%\cos 2a & = \cos a - \sin a =
%\nonumber \\
%& = 2\cos a - 1.
%\end{align}
%\end{example}
%\end{enumerate}

% Пунктуация в формулах
%\item
%Команды принудительного перехода на новую страницы \newpage,
%\clearpage и др. не используются. Совсем и никогда.
%\item
%Команды принудительного переноса строки \textbackslash, \newline и др.
%используются только в заголовках, коло
%
%Кавычки
%
%«»
%
%сsquotes
%
%\texttt{\textasciitilde{}-\/-\/-}
%
%\eqref
%
%is/are vs \texttt{\textasciitilde{}-\/-\/-}
%
%\texttt{\textbackslash{}quad} в формулах
%\cdots vs \ldots
% graphics vs graphixs
% Англоязычная традиция
\end{document}