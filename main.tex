% !TeX encoding = UTF-8
\documentclass[12pt, a4paper]{article}
\usepackage[utf8]{inputenc}
\usepackage[english, russian]{babel}

% for indent first paragraph within section
\usepackage{indentfirst}

%for text boundaries
\usepackage[a4paper, left=15mm, right=15mm, top=15mm, bottom=15mm]{geometry}

%for some mathematical things
\usepackage{amsmath}

\usepackage{listings}

\usepackage{tgpagella}
\usepackage{dtk-logos}

\usepackage{sverb}

\title{{\LaTeXe} Best Practices}
\author{Алексей Ведяков}
\date{2020}

\begin{document}

\maketitle
 
Ниже изложены основные практики, которые настоятельно рекомендуются к
применению, а в некоторых случаях являются обязательными. Их применение
обусловлено основами типографики, традициями сложившимися в русской и
англоязычной академической среде, здравым смыслом, опытом работы в TeX, в том числе совместной, и, конечно, в некоторой степени субъективными причинами.

\section{Оформление формул}

\subsection{Окружение}

Все блочные формулы только в окружении \texttt{align}. Без
\texttt{aligned}, \texttt{equation}, \texttt{\$\$\ldots\$\$} и так далее, и уже
тем более без вложенного \texttt{aligned} в \texttt{align}.

\textbf{Причина:} \texttt{align} --~самое мощное окружение, другие нет
причин использовать, а унификация помогает сделать оформление формул в
документе единообразным, уменьшить проблемы при совместной работе.

\subsection{Нумерация формул}

%\begin{enumerate}
%	\item Для выравнивания многострочных формул используется \& (амперсанд)
%	\begin{enumerate}
%		\item Знак ставится до математического знака, иначе оно <<прилипает>> к стоящему справа выражению:
%		
%		\begin{minipage}{\linewidth}
%\begin{example}[side-by-side]
%	\begin{align}
%		y(t) = & ax(t) + b. % wrong
%	\end{align}
%\end{example}%
%\vspace{-3em}%
%\begin{example}[side-by-side]
%	\begin{align}
%		y(t) & = ax(t) + b.
%	\end{align}
%\end{example}
%		\end{minipage}
%		
%		\item Формулы выравниваются по знаку равно:
%		
%		\begin{minipage}{\linewidth}
%\begin{example}[side-by-side]
%	\begin{align}
%		y(t) & = a_1 x(t) + b_1,
%		\\
%		z(t) & = a_2 x(t) + b_2.
%	\end{align}
%\end{example}
%		\end{minipage}	
%	\end{enumerate}
%	\item Все формулы и строки в формулах нумерованные, вручную нумерация убирается только с помощью команды \nonumber для строк, которые являются промежуточными в вычислениях. Например, 
\begin{demo}[narrow]{}
	\begin{align}
	\cos 2a & = \cos a - \sin a \nonumber \\
		& = 2\cos a - 1
\end{align}
\end{demo}
%	В таком случае нумеруется только последняя строчка.
%
%\end{enumerate}

%\item
%Команды принудительного перехода на новую страницы \newpage,
%\clearpage и др. не используются. Совсем и никогда.
%\item
%Команды принудительного переноса строки \textbackslash, \newline и др.
%используются только в заголовках, коло
%
%Кавычки
%
%«»
%
%сsquotes
%
%\texttt{\textasciitilde{}-\/-\/-}
%
%\eqref
%
%is/are vs \texttt{\textasciitilde{}-\/-\/-}
%
%\texttt{\textbackslash{}quad} в формулах
%\cdots vs \ldots
% graphics vs graphixs
\end{document}